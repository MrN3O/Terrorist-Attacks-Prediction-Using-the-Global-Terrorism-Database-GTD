\documentclass[conference]{IEEEtran}
\IEEEoverridecommandlockouts
% The preceding line is only needed to identify funding in the first footnote. If that is unneeded, please comment it out.
\usepackage{cite}
\usepackage{amsmath,amssymb,amsfonts}
\usepackage{graphicx}
\usepackage{textcomp}
\usepackage{xcolor}
\usepackage{algorithm}
\usepackage{algpseudocode}
\usepackage{multirow}
\def\BibTeX{{\rm B\kern-.05em{\sc i\kern-.025em b}\kern-.08em
    T\kern-.1667em\lower.7ex\hbox{E}\kern-.125emX}}
\begin{document}

\title{Prediction of Terrorist Attacks over the Globe Using the Global Terrorism Database: A Comparative Analysis of Machine Learning Prediction Algorithms\\
{\footnotesize \textsuperscript}
\thanks{}
}

\author{\IEEEauthorblockN{1\textsuperscript{st} Happy}
\IEEEauthorblockA{\textit{School of Computer science and Engineering} \\
\textit{Lingayas Vidyapeeth, Faridabad}\\
Haryana, 121002, India \\
dr.hpyrjpt@gmail.com}
\and
\IEEEauthorblockN{2\textsuperscript{nd} Manoj Yadav}
\IEEEauthorblockA{\textit{School of Computer science and Engineering} \\
\textit{Lingayas Vidyapeeth, Faridabad}\\
Haryana, 121002, India \\
manoj200.yadav@gmail.com}
}
\maketitle

\begin{abstract}
Terrorism is a means by which some group of people or organization impacts humankind by the unlawful use of violence and extortion against the government, Ideology, process, and civilians to pursue their political, religious, and economic aim. Terrorism is the most significant critical risk for the governments and the civilians of a particular geographical location.

Identifying whether the attack is a terrorist attack or not within a short period itself is a big problem for security forces. Predication of Terrorist groups, weapons type, attack type, Target, and organization behind that terrorist attack helps the security forces minimize the risk and save multiple lives. Many previous studies have used a worldwide terrorism data set to examine the terrorist assaults. Many researchers have also solved the projection of past unsolved terrorist attack problems. 

This paper mainly focused on predicting future terrorist attacks using machine learning prediction algorithms and identifying whether the attack is a terrorist attack or not, along with the severity of the attack, weapons type, attack type, and Targets. Prediction of Terrorist attacks reduces the loss of lives in terrorist attacks. It also helps law enforcement agencies to act accordingly and minimize the risk as much as possible.
\end{abstract}

\begin{IEEEkeywords}
Terrorist Attack, Global Terrorism Dataset(GTD), Machine Learning algorithms, Attack predication, Data visualization, Data pre-processing, Data Analysis.
\end{IEEEkeywords}

\section{Introduction}
The terrorist attack is one of the most significant threats to that particular geographical area's government. The Institute for Economics and Peace (IEP) has released its annual study related to the Global Terrorism Index (GTI), 2020 [8] states that a total of 13,826 people lost their lives with an Impact of Property Damage of around \$1,777.60 million due to terrorism all over the world even though it fell by 15.5 percent for the 5th following year after reaching a new peak in 2014.

The Global Terrorism Database (GTD) [5] is one of the major widespread unclassified, open-source online databases. The National Consortium for the Study of Terrorism and Responses to Terrorism (START) manages this database. The same is available for research for both individuals and industries.

As per the necessity of the demand for future terrorist attack prediction systems, most research scholars solve the problem based on historical data analysis of terrorist attacks. The prior research mainly focuses on three aspects of GTD. One is predicting the null values available in the global terrorism database, like the Organization responsible for that attack. The second is to predict the future event by numerous machine-learning and deep learning models. The third is to predict the behavior and the relationship between the terrorist organizations.

This research paper was further structured into the following sections: a literature review of related work available in this context in section II. The research methodology explains the data set, data pre-processing steps, used tools, and purposed system details are available in Section III. Section IV contains Results having data analysis, prediction, and comparison of machine learning prediction algorithms. Section V covered the conclusion and Ideas related to further work is covered in section VI.

\section{Literature Review}
The evolution of terrorism, which is more brutal day by day with the help of typical technology changes, left a terrible impact on society with lots of long-term effects. It attracts many researchers to produce better solutions with advanced technology and other facilities, researchers' main aim is to predict the attack as soon as possible and reduce the casualties to the maximum.

Sattar et al., 2021 \cite{b14} uses deep learning models for Proposal and Enactment of recommender System for Iraqi Terrorist attack Database Based, a subset of GTD. Researchers use the proposed system's restricted Boltzmann Machine (RBM) neural Network to predict terrorist attacks.

Pan, 2021 \cite{b13} proposed a framework containing data splitting, pre-processing, and different five prediction models. Researchers use sci-kit learn and Select K best for feature selection. further, they use different five kinds of machine learning prediction algorithm, which produces a significantly better result than earlier proposed systems. Olabanjo et al., 2021 \cite{b10} proposed the assembled ML model, which cartels SVM and kNN for the prediction of areas prone to attack by terrorist groups.

Huamaní et al., 2020 \cite{b7} describe different Machine Learning techniques to predict Terrorist attacks over the globe. They represent the many other author’s techniques in their paper. They also explain different artificial intelligence techniques like symbolic AI, Evolutionary AI, and Machine learning Techniques.

Zhenkai et al., 2020 \cite{b19} designed a system that uses the various Machine learning methods and is used for terrorist attack risk predictions. Bhatia et al., 2020 \cite{b3} use Hadoop on GTD. They further use the HiveQL to retrieve essential evidence around the attack. Further, they use Tableau to visualize the data set.

Agarwal et al., 2019 \cite{b1} uses the Global Terrorism Database (GTD) to predict the effectiveness of the violence by using SVM, RF, and Dummy Classifier. Here, researchers used the weather data set to find out the relation between weather conditions and terrorist attack patterns. Researchers Merged the GTD with the Weather Data formed and pre-processed the new combined data set, then visualization and analysis on the new collective data set. The output of the previous step is further processed to predict successfulness and prediction of causalities based on weather conditions.

Singh et al., 2019 \cite{b15} proposed a system that analysis terrorist countries and regions with other knowledge.They try to predict terrorist behavior and uses different Machine learning techniques. Gao et al., 2019 \cite{b4} also described five different machine learning techniques for terrorist attack prediction. 

Spiliotopoulos et al., 2019 \cite{b16} proposed a system to predict the terrorist attack risk factor in a particular geographical area using Global Terrorism Database data from 2000 to 2017.

Xia \& Gu, 2019 \cite{b18} proposed a model called Terrorist knowledge graph (TKG). This graph is updated regularly for better prediction by extracting information from the Wikipedia page. The TKG also provides a better understanding of both humans and machines to produce results.

Alhamdani et al., 2018 \cite{b2} provide a review of the recommendation system of GTD using Deep learning techniques to uncover the social media propaganda for terrorism using the Global Terrorism Database.

Kolajo \& Daramola, 2017 \cite{b9} proposed a system model which uses different social media sources to identify the terrorist’s activities using Apache Spark technology to achieve the desired results. Zijuan \& Shuai, 2017 \cite{b20} use big data analysis methods to study terrorist attacks and analyze strategies to review the data related to previous terrorist attacks.
 
Toure \& Gangopadhyay, 2016 \cite{b17} solve the various aspect of terrorism with the help of various software and methodologies. Using this system, they aim to calculate risks in different locations and risks in near future terrorist attacks. Hegde et al., 2016 \cite{b6} proposed their work on visual analytics on GTD with the current events of social media, i.e., Social Network analytics.

Pagan, 2010 \cite{b12} analyzed different pre-processing techniques to reduce the noise in the data. Further they used other machine learning techniques to minimize the error and enhance the accuracy of the system. Ozgul et al., 2009 \cite{b11} proposed a model to solve various terrorist assaults which were unsettled in the global terrorism database (GTD). Using the clustering method, they used the Crime prediction model (CPM) to predict unsolved attacks in Turkey between 1970 and 2005.

\section{Research Methodology}
This section is divided further into the following sub-sections that describe the methodology used for carrying out the purposed work:
\subsection{Dataset}
For the proposed work, Global Terrorism Dataset (GTD) is used [5]. GTD is an open-source repository based on terrorist incidences from 1970 to 2020 and is updated periodically. National Consortium is responsible for regularly updating, maintaining, and studying terrorists and answers to terrorism (START) at Maryland University [8]. This database consists of more than 2,01,183 terrorist incidence information and is updated every year. These Incidence’s further labelled by more than 135 attributes or columns to dissipate the possibilities and totality of the event. Various essential attributes of the GTD are as follows:
\begin{enumerate}
\item \emph{iyear, imonth, iday:} Year, month, and date respectively in the numeric form on which terrorist attack happened.
\item \emph{country, country\_txt, city, latitude, longitude:} Country Code, Country Name, city name in which the attack happened, and coordinates in the form of latitude and longitude for that location. 
\item \emph{crit1, crit2, crit3:} crit1, crit2, and crit3 are binary numerical values that identify the purpose of the terrorist attack.
\item \emph{attacktype1\_txt:} This attribute contains the details of attack categories example: Hijacking, Bombing/Explosion, and many more.
\item \emph{weaptype1\_txt:} Kind of firearm used in the violence. Example Explosives, Fire arms, Incendiary and many more.
\item \emph{nkill, nkillter:} nkill counts the total number of people who die in that particular incidence while nkillter only terrorist number count who die in that particular incidence.
\item \emph{Nwound:} Number count of confirmed non-fatal injuries for both victims and terrorists.
\end{enumerate}

\begin{figure}[htbp]
        \centering\centerline{\includegraphics[width=1\columnwidth]{fig2_A.jpg}}
        \caption{Sample of GTD Before Data Pre-processing}
        \label{fig}
        \end{figure}
        
A detailed description of all the one hundred and thirty-five attributes can be found in the GTD Codebook of studying Terrorists and Responses to Terrorism, ID: 231483. This Document was initially uploaded on 7 September 2019 and Updated in August 2021.

\subsection{Data pre-processing}
Preprocessing of the GTD database contains the following steps:

\begin{enumerate}
    \item \emph{Dropping irrelevant columns:} The Global Terrorism Database has 2,01,183 records or incidents, and 135 attributes or columns further describe each record or incident. Out of these, most attributes are irrelevant based on the user's requirement, so some of the attributes or columns were dropped in the pre-processing step. The irrelevant columns were dropped based on the following criteria.
    \begin{enumerate}
        \item \emph{Based on Percentage of Null Entries: } Remove columns with more than 50\% (percent) of Null Values. The null entry Percentage can be calculated using the following equation. 
        \begin{equation}
        Null Entries(\%)=\frac{Total Null Entry}{Total Entries}*100\label{eq}
        \end{equation}
        
        \emph{Example:} If any column or attribute of GTD has more than 1,00,592 null values, that attribute or column is eternally removed from the dataset.
        
        \item \emph{Based on the User Requirement:  } Remove the columns that are not fit as per the user's desires.
        
    \end{enumerate}

\item \emph{Renaming the columns:} Column or attribute names in the GTD were also renamed for ease of work. These changes help in better understanding and reduce the confusion at the time of Data analysis. Like 'gname' rename to 'Group\_Name', 'targtype1\_txt' to 'Targtype\_Name' and others. 
\item \emph{Replacing unknown values:} We use the most frequent item method to replace null values. We replace unknown Cities, Attack Types, and Target Types with the most frequent Cities, Attack Types, and Target Types in that specific country, along with replacing unknown Group Names and Weapon Types with the most frequent Group Names and Weapon Types in that specific Region. Pseudo code for replacing unknown values is shown in algorithm 1.   
\item \emph{Handling irrelevant values:}  Even after removing null values, some of the GTD columns/ attributes still have irrelevant values in the form of ‘-’ (negative) values like the 'doubtter' attribute or '0' (zero) in the 'iday,' 'imonth,' and 'iyear.' For the 'doubtter' negative values were directly replace with '1' because they most likely to be true for this attribute. The 'iday,' 'imonth,' and 'iyear' unknown values were replaced with the help of most frequently day, month, and year, respectively. 
\item \emph{Dropping the duplicate and missing values rows:} Finally, we drop the duplicate and missing values from the GTD to remove the redundancy and data inconsistency.
\item \emph{Terrorist Attack Criteria:} One of the fundamental and most significant problems in any attack for law enforcement agencies is to identify an attack is a terrorist attack or not. To differentiate between attacks, we use 'Crit1', 'Crit2', and 'Crit3.' as the different motivations behind the terrorist attack.\\

\begin{algorithm}
    \caption{Replace unknown values with most frequent}\label{alg:cap}
    \begin{algorithmic}[1]
     \Require $replace(i,j)$\Comment{Replace Function}
     \State $create \gets [City, AttackType\_Name, Targtype\_Name]$
     \State $df \gets gtd.csv$
        \For {$j = create[0]$ to $create[2]$}
            \For {$i = 0$ to $len(df[j])$}
                \If {($df[j][i] == 'Unknown'$)}
                    \State $df[j][i] = replace(i,j)$
                \EndIf
            \EndFor
        \EndFor
    \end{algorithmic}
\end{algorithm}
\end{enumerate}

\begin{figure}[htbp]
        \centering\centerline{\includegraphics[width=1\columnwidth]{fig2_B.jpg}}
        \caption{Sample of GTD after Data Pre-processing}
        \label{fig}
        \end{figure}

\subsection{Jupyter lab and Orange Tool}
After executing all the pre-processing steps on the GTD, cleaned data was saved in CSV format. The same was passed to the Jupyter lab \& Orange tool for further Data Analysis and machine learning predictions. Jupyter lab is a web-based interface with an updated version of the jupyter notebook. Orange Tool kit is used for data mining, data visualization, and machine learning. Jupyter lab and jupyter notebook both are part of the anaconda navigator.

\subsection{Purposed System}
The proposed system in this paper is mainly focused on Data analysis and visualization using existing historical data to uncover information, such as year-wise attack count, a prime target, attacking methods, region-wise terrorist activity, most affected countries list, and others along with the prediction of future terrorist attacks using various machine learning algorithms such as kNN, Naive Bayes, Neural Network, Logistic Regression, Random forest, Tree, Cn2 Rule inducer, and SVM.

Machine learning algorithms are further used to predict the severity of terrorist attacks based on the consolidated yield of data analysis and visualisation.

Further accuracy of the proposed system for the different machine learning algorithms is compared to find the best suitable algorithm for this problem, which is further compared with the existing systems to evaluate enhancement achieved using the proposed approach. 
\begin{figure}[htbp]
        \centering\centerline{\includegraphics[width=1\columnwidth]{fig1.png}}
        \caption{System Architecture}
        \label{fig}
        \end{figure}
 
The system architecture shown in fig 3 represents the methodology used to achieve this desired task in diagrammatic representation to carry out the purposed work. The pseudo-code for one of the machine learning algorithms used to achieve the task, i.e., kNN for training and Implementing the GTD, is shown in Algorithm 2.
\begin{algorithm}
    \caption{Training and Implemented kNN for GTD}\label{alg:cap}
\begin{algorithmic}[1]
    \Function{Train\_kNN}{D, GTD}
             \State $GTD' \gets preprocessed(GTD)$
             \State $k \gets select\_k(D,j)$
             \State \Return $GTD', K$
    \EndFunction
    
    \Function{Implemented\_kNN}{D, GTD', k, d}
                 \State $S_k \gets Calculate\_Nearest\_Neighbors(GTD', k, d)$
                    \For{\texttt{\textbf{every} $D_i \epsilon D$}}
                        \State $p_j \gets |s_k \cap D_j|/k$
                    \State \Return $arg\_max, p_j$
                    \EndFor
    \EndFunction
\end{algorithmic}
\end{algorithm}

\section{Result}
This segment contains numerous observations, including data visualization, prediction of terrorist attacks and severity of attack and comparisons between various machine learning prediction algorithms. 

In the initial findings, fig. 4 shows that the number of terrorist attacks in 2014 reached the new maximum from 1970 to 2019. 
\begin{figure}[htbp]
        \centerline{\includegraphics[width=1\columnwidth]{fig3.jpg}}
        \caption{Year-wise Terrorist attacks count}
        \label{fig}
        \end{figure}

Along with this, fig. 5 revealed that Bombing/Explosions, Armed assault, and assassination are the top three attacking methods used by terrorists.
\begin{figure}[htbp]
        \centerline{\includegraphics[width=1\columnwidth]{fig4.jpg}}
        \caption{Attacking Methods used by Terrorists}
        \label{fig}
        \end{figure}

Fig. 6 shows that during 1970 to 2009 period, private citizens and property were prioritized in any attack, followed by the military, police, and Government (General). People who belong to these occupations or groups are on the hit list of attackers.
\begin{figure}[htbp]
        \centerline{\includegraphics[width=1\columnwidth]{fig5.jpg}}
        \caption{Main Targets of Terrorist}
        \label{fig}
        \end{figure}

The data set also clearly shows in fig. 7 that after 2010, North Africa and the Middle East were heavily impacted by terrorists, followed by Eastern Europe. 
\begin{figure}[htbp]
        \centerline{\includegraphics[width=1\columnwidth]{fig6.jpg}}
        \caption{Year-wise terrorist activities in various region of the world}
        \label{fig}
        \end{figure}

In the fig. 8 analysis, shows that Iraq is the utmost impacted nation by terrorism, followed by Pak and Afghanistan, where as India is the fourth-most affected country by terrorism.
\begin{figure}[htbp]
        \centerline{\includegraphics[width=1\columnwidth]{fig7.jpg}}
        \caption{Most affected Counties by Terrorism}
        \label{fig}
        \end{figure}

Visual Representation in fig. 9 clearly shows that in South Asia, facility/ Infrastructure attacks followed by armed attacks were the primary weapon used by the terrorists. When it comes to the terrorist activities done by particular groups. Fig. 10 shows that Taliban became the most active terrorist worldwide after the year 2000.
\begin{figure}[htbp]
        \centerline{\includegraphics[width=1\columnwidth]{fig8.jpg}}
        \caption{Country wise attack type.}
        \label{fig}
        \end{figure}

\begin{figure}[htbp]
        \centerline{\includegraphics[width=1\columnwidth]{fig9.jpg}}
        \caption{Year wise terrorist group activity.}
        \label{fig}
        \end{figure}

Visual representation in fig. 11 reveals the pattern between attacks vs. killed in a particular country. 

\begin{figure}[htbp]
        \centerline{\includegraphics[width=1\columnwidth]{fig10.jpg}}
        \caption{Country wise attack vs killed.}
        \label{fig}
        \end{figure}

Further fig. 12 shows that the May, April, and August months are the most exposed months with the highest risk ratio, i.e., 10.05, 9.68, and 9.25, respectively, for terrorist attacks and the 25\textsuperscript{th} and 15\textsuperscript{th} of every month are the riskiest for a terrorist attack compared to other days of the month. The day 25\textsuperscript{th} of the month has the highest risk ratio, i.e., 3.67, followed by date 15\textsuperscript{th} with a 3.61 risk ratio as show in the fig. 13.

\begin{figure}[htbp]
        \centerline{\includegraphics[width=.3\columnwidth]{fig11.jpg}}
        \caption{Month-wise terrorist attack risk.}
        \label{fig}
        \end{figure}

Further, for the prediction of whether terrorist attacks are successful or not, the terrorist attack prediction model created in the Orange Data mining tool is shown in fig. 14.

The Terrorist attack prediction model produces the results based on various machine learning prediction algorithms such as Random Forest, Neural Network, Naive Bayes, Logistic Regression, KNN, CN2 rule inducer, Tree, and SVM.

\begin{figure}[htbp]
        \centerline{\includegraphics[width=.6\columnwidth]{fig12.jpg}}
        \caption{Day-wise terrorist attack risk.}
        \label{fig}
        \end{figure}

For the same AUC (Area under ROC Curve), CA (Classification Accuracy), F1, Precision and Recall for attack prediction and attack are successful or not are as shown in fig. 15. 

\begin{figure}[htbp]
        \centerline{\includegraphics[width=1\columnwidth]{fig13.jpg}}
        \caption{Attack Predication Model.}
        \label{fig}
        \end{figure}

The Tree representation of attack Severity prediction of successful terrorist attacks using machine learning Tree algorithms is shown in fig 16. The severity of a terrorist attack is successfully predicted with the accuracy of 89.98 percent using the Neural Network algorithm with others as shown in fig. 17.  
\begin{figure}[htbp]
        \centerline{\includegraphics[width=1\columnwidth]{fig14.jpg}}
        \caption{Attack prediction accuracy matrix}
        \label{fig}
        \end{figure}

\begin{figure}[htbp]
        \centerline{\includegraphics[width=1\columnwidth]{fig15.jpg}}
        \caption{Tree representation for attack severity prediction}
        \label{fig}
        \end{figure}

\begin{figure}[htbp]
        \centerline{\includegraphics[width=1\columnwidth]{fig16.jpg}}
        \caption{Attack severity prediction accuracy matrix}
        \label{fig}
        \end{figure}

When the proposed system output is compared with the existing available systems, it comes under notice that the proposed system reveals some very critical information regarding the terrorist attacks that happened in the past. It also produces a significantly better result for the prediction of Terrorist attacks compared to the already available systems that use the same data set. The existing model recommended by Sattar et al., 2021 \cite{b14} predicts with the accuracy of 98.12\%, Pan, 2021 \cite{b13} proposed system predicting the terrorist attacks within the range of accuracy between 97.16\% and 96.82\%, Olabanjo et al., 2021 \cite{b10} purposed hybridized classical gave an accuracy of 97.81\% while Huamaní et al., 2020 \cite{b7} can predict the terrorist attack with an accuracy between 75.45\% to 90.414\%, which is significantly low compared to the proposed system. With the help of better pre-processing techniques and attribute selections, the proposed system can produce better prediction results with an accuracy of 98.2 and 89.8 percentile for the prediction of attack and successfulness or severity of an attack, respectively.

Table I compares the proposed system with the existing systems that use different methods and practices on the Global Terrorism Database.

\begin{table}
%\centering
\caption{ A comparison of intended systems work on GTD}
\begin{center}
\begin{tabular}{|l|l|c|}
\hline
\multicolumn{1}{|c|}{\textbf{\begin{tabular}[c]{@{}c@{}}Related Model Purposed\\  By\end{tabular}}} & \multicolumn{1}{c|}{\textbf{\begin{tabular}[c]{@{}c@{}}Related Model \\ Accuracy (\%)\end{tabular}}} & \textbf{\begin{tabular}[c]{@{}c@{}}Purposed work\\  Accuracy (\%)\end{tabular}} \\ \hline
Sattar et al., 2021 {[}14{]}                                                                        & 98.12\%                                                                                              & \multirow{8}{*}{98.20\%}                                                        \\ \cline{1-2}
Pan, 2021 {[}13{]}                                                                                  & 96.82\%-97.16\%                                                                                      &                                                                                 \\ \cline{1-2}
Olabanjo et al., 2021{[}10{]}                                                                       & 97.81\%                                                                                              &                                                                                 \\ \cline{1-2}
Huamaní et al., 2020 {[}7{]}                                                                       & 75.45\%-90.414\%                                                                                     &                                                                                 \\ \cline{1-2}
Singh et al., 2019 {[}15{]}                                                                         & 82\%                                                                                                 &                                                                                 \\ \cline{1-2}
Agarwal et al., 2019 {[}1{]}                                                                        & 68\%-82\%                                                                                            &                                                                                 \\ \cline{1-2}
Toure et al., 2016 {[}17{]}                                                                         & 96.3\%                                                                                               &                                                                                 \\ \cline{1-2}
Gao et al., 2019 {[}4{]}                                                                            & 94.8\%                                                                                               &                                                                                 \\ \hline
\end{tabular}
\end{center}
\end{table}

\section{Conclusion}
This research paper we compares eight machine learning prediction algorithms for the proposed system, including Random Forest, Neural Network, Naive Bayes, Logical regression, kNN, CN2 Rule inducer, Tree, and SVM. This experiment result shows that  Random forest, Neural Network, Naive Bayes, kNN, and Logical regression have the highest Classification Accuracy (CA), up to 98.2\%, followed by Logical regression and CN2 Rule inducer, which have the same classification accuracy but low AUC. Here, SVM produces the least significant result with the accuracy of 7.9\% and 50\% for the Area Under the Curve (AUC).

For the prediction of attack severity, Random forest produces the best results with the accuracy of 89.8\% and AUC with 50.5\%, while SVM again produces the worst accuracy for the prediction of attack severity with 50.10\%.  

With the help of data visualization and machine learning prediction algorithms, lots of shocking and vital information is disclosed, giving information related to attack patterns, weapons used, and most vulnerable locations and months and days. It has been observed that May 25\textsuperscript{th} and 15\textsuperscript{th} have the most vulnerable day in a calendar year. It also shows that every terrorist group has different geographical areas in which they operate. The topmost terrorist groups also operate in different geographical locations and never operate in each other group's areas. Other facts also reveal that the Taliban and PKK have a perfect attack pattern that makes them almost unpredictable. Our Prediction system also predicts future terrorist attacks, one of law enforcement agencies' basic but essential problems.

\section{Future Work}

Future work will focus on predicting group alliance possibilities. The attack type and location may also be predicted based on social media data like Facebook and Twitter. Local News agencies' reports, weather, and special occasion’s data may also help to predict the attacks with faster and better accuracy with the prediction of severity of that attack.

Future work may also focus on developing a better prediction system by using better and enhanced prediction algorithms with smart preprocessing techniques. 


\begin{thebibliography}{00}
\bibitem{b1} Agarwal, P., Sharma, M., \& Chandra, S. (2019). Comparison of Machine Learning Approaches in the Prediction of Terrorist Attacks. 2019 Twelfth International Conference on Contemporary Computing (IC3), 1–7. https://doi.org/10.1109/IC3.2019.8844904

\bibitem{b2} Alhamdani, R. S., Abdullah, M. N., \& Sattar, I. A. (2018). Recommender System for Global Terrorist Database Based on Deep Learning. International Journal of Machine Learning and Computing, 8(6), 6

\bibitem{b3} Bhatia, K., Chhabra, B., \& Kumar, M. (2020). Data Analysis of Various Terrorism Activities Using Big Data Approaches on Global Terrorism Database. 2020 Sixth International Conference on Parallel, Distributed and Grid Computing (PDGC), 137–140. https://doi.org/10.1109/PDGC50313.2020.9315784

\bibitem{b4} Gao, Y., Wang, X., Chen, Q., Guo, Y., Yang, Q., Yang, K., \& Fang, T. (2019). Suspects Prediction towards Terrorist Attacks Based on Machine Learning. 2019 5th International Conference on Big Data and Information Analytics (BigDIA), 126–131. https://doi.org/10.1109/BigDIA.2019.8802726

\bibitem{b5} Global Terrorism Database. Start.umd.edu. (2021). Retrieved 16 September 2021, from https://www.start.umd.edu/gtd/access. 

\bibitem{b6} Hegde, L. V., Sreelakshmi, N., \& Mahesh, K. (2016). Visual Analytics of Terrorism Data. 2016 IEEE International Conference on Cloud Computing in Emerging Markets (CCEM), 90–94. https://doi.org/10.1109/CCEM.2016.024

\bibitem{b7} Huamaní, E. L., Mantari, A., \& Roman-Gonzalez, A. (2020). Machine Learning Techniques to Visualize and Predict Terrorist Attacks Worldwide using the Global Terrorism Database. International Journal of Advanced Computer Science and Applications, 11(4). https://doi.org/10.14569/IJACSA.2020.0110474

\bibitem{b8} Institute for Economics \& Peace. (2021). Global terrorism index 2020 measuring the impact of terrorism [Ebook] (p. 109). Retrieved 16 September 2021, from https://www.economicsandpeace.org/wp-content/uploads/2020/11/GTI-2020-web-2.pdf.

\bibitem{b9} Kolajo, T., \& Daramola, O. (2017). Leveraging big data to combat terrorism in developing countries. 2017 Conference on Information Communication Technology and Society (ICTAS), 1–6. https://doi.org/10.1109/ICTAS.2017.7920662

\bibitem{b10} Olabanjo, O. A., Aribisala, B. S., Mazzara, M., \& Wusu, A. S. (2021). An ensemble machine learning model for the prediction of danger zones: Towards a global counter-terrorism. Soft Computing Letters, 3, 100020. https://doi.org/10.1016/j.socl.2021.100020

\bibitem{b11} Ozgul, F., Erdem, Z., \& Bowerman, C. (2009). Prediction of past unsolved terrorist attacks. 2009 IEEE International Conference on Intelligence and Security Informatics, 37–42. https://doi.org/10.1109/ISI.2009.5137268

\bibitem{b12} Pagan, J. V. (2010). Improving the classification of terrorist attacks a study on data pre-processing for mining the Global Terrorism Database. 2010 2nd International Conference on Software Technology and Engineering, 5608902. https://doi.org/10.1109/ICSTE.2010.5608902

\bibitem{b13} Pan, X. (2021). Quantitative Analysis and Prediction of Global Terrorist Attacks Based on Machine Learning. Scientific Programming, 2021, 1–15. https://doi.org/10.1155/2021/7890923

\bibitem{b14} Sattar, I. A., Alhamdani, R. S., \& Abdulla, M. N. (2021). Design and Implementation Recommender System for Iraqi Terrorist Database Based on Deep Learning. 2021 7th International Engineering Conference “Research \& Innovation amid Global Pandemic" (IEC), 32–36. https://doi.org/10.1109/IEC52205.2021.9476083


\bibitem{b15} Singh, K., Chaudhary, A. S., \& Kaur, P. (2019). A Machine Learning Approach for Enhancing Defence against Global Terrorism. 2019 Twelfth International Conference on Contemporary Computing (IC3), 1–5. https://doi.org/10.1109/IC3.2019.8844947

\bibitem{b16} Spiliotopoulos, D., Vassilakis, C., \& Margaris, D. (2019). Data-driven country safety monitoring terrorist attack prediction. Proceedings of the 2019 IEEE/ACM International Conference on Advances in Social Networks Analysis and Mining, 1128–1135. https://doi.org/10.1145/3341161.3343527

\bibitem{b17} Toure, I., \& Gangopadhyay, A. (2016). Real time big data analytics for predicting terrorist incidents. 2016 IEEE Symposium on Technologies for Homeland Security (HST), 1–6. https://doi.org/10.1109/THS.2016.7568906

\bibitem{b18} Xia, T., \& Gu, Y. (2019). Building Terrorist Knowledge Graph from Global Terrorism Database and Wikipedia. 2019 IEEE International Conference on Intelligence and Security Informatics (ISI), 194–196. https://doi.org/10.1109/ISI.2019.8823450

\bibitem{b19} Zhenkai, L., Yimin, D., \& Jinping, L. (2020). Analysis Model of Terrorist Attacks Based on Big Data. 2020 Chinese Control and Decision Conference (CCDC), 3622–3628. https://doi.org/10.1109/CCDC49329.2020.9164626

\bibitem{b20} Zijuan, L., \& Shuai, D. (2017). Research on Prediction Method of Terrorist Attack Based on Random Subspace. 2017 International Conference on Computer Systems, Electronics and Control (ICCSEC), 320–322. https://doi.org/10.1109/ICCSEC.2017.8446815
\end{thebibliography}

\end{document}
